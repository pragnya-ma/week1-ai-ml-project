\documentclass{article}

\title{Week 1 Project Report}
\author{Pragnya}
\date{10 December 2025}

\begin{document}

\maketitle

\section{Introduction}
This report documents the progress made in Week 1 of the AI/ML project.
Week 1 was focused on understanding basic python tools required for the project.

\section{Data Search}
For this project, I selected a real-world AQI dataset from kaggle which contains hourly air quality measurements for Indian cities over multiple years, together with basic meteorological variables. This dataset will be used in later tasks for data analysis and builidng simple prediction models.

The main dataset used in Week1 is:
\begin{itemize}
    \item \textbf{Source:} Kaggle -- "Air Quality Dataset: Indian Cities (2022-2025)" by Bhautik Vekariya.
    \item \textbf{File used in this project:} \texttt{INDIA\_AQI\_COMPLETE\_20251126.csv}.
    \item \textbf{Time period:} 2022-2025, with \textbf{hourly} observations.
    \item \textbf{Cities covered:} 29 major Indian cities, including Delhi, Mumbai, Bengaluru, Chennai, and Hyderabad.
    \item \textbf{Main variables:} AQI, PM\textsubscript{2.5}, PM\textsubscript{10}, NO\textsubscript{2}, SO\textsubscript{2} CO, O\textsubscript{2}, pls temperature, humidity, and wind related features.
    \item \textbf{Approximate size:} around 8.4. lakh hourly records.
\end{itemize}

\section{Methods}
Details about tools installed, setup, and workflow.
I installed python and its libraries(numpy, pandas, matplotlib, scikit-learn, jupyter), created and activated a virtual environment and learned how to use VS code.

\section{Results}
Summary of what worked so far.
\textbf{Dataset successfully loaded in Python.}
Initial inspection using \texttt{pandas}
No critical errors during import

\section{Discussion}
Next steps and Reflections.
Next steps include feature selection, visualization, and modeling.
\end{document}